\documentclass{article}

%\usepackage{graphicx}
%\usepackage{float}
\usepackage[letterpaper, margin=1in]{geometry}
\usepackage{amsmath}
\usepackage{listings}
\usepackage{lstautogobble}
\usepackage{xepersian}

%\graphicspath{{./img}}
\settextfont{XB Niloofar}

\title{طراحی الگوریتم - دکتر قوامی‌زاده}
\author{امیرحسین منصوری - ۹۹۲۴۳۰۶۹}
\date{تمرین سری دوم}

\allowdisplaybreaks

\begin{document}
	\maketitle

	\section*{سوال ۱}
	\subsection*{الف)}
	\paragraph*{}

	می‌دانیم وجود دارد
	$c_1, c_2, c_3, c_4, n_0, n_1$
	به قسمی که

	\begin{align*}
		\forall n &\ge n_0, c_1g(n) \le f(n) \le c_2g(n) \\
		\forall n &\ge n_1, c_3h(n) \le g(n) \le c_4h(n)
	\end{align*}

	نامساوی‌های بالا به ازای همه
	\begin{align*}
		n \ge n_2 &= \max(n_0, n_1)
	\end{align*}

	نیز درست است. داریم

	\begin{align*}
		&f(n) \le c_2g(n) \le c_2c_4h(n) \; (\forall n \ge n_2) \\
		&f(n) \ge c_1g(n) \ge c_1c_3h(n) \; (\forall n \ge n_2) \\
		\Rightarrow
		c_1c_3h(n) \le &f(n) \le c_2c_4h(n) \; (\forall n \ge n_2)
	\end{align*}

	تعریف می‌کنیم
	$c_5 = c_1c_3, c_6 = c_2c_4$.
	بنابراین
	\begin{equation*}
		c_5h(n) \le f(n) \le c_6h(n) \; (\forall n \ge n_2)
	\end{equation*}

	و بنابراین
	$f(n) \in \Theta(h(n))$.

	\subsection*{ب)}
	\paragraph*{}

	طرف اول را اثبات می‌کنیم. می‌دانیم وجود دارد
	$c_1, c_2, n_0$
	به قسمی که

	\begin{gather*}
		c_1g(n) \le f(n) \le c_2g(n) \; (\forall n \ge n_0) \\
		c_1, c_2, n_0 > 0
	\end{gather*}

	داریم

	\begin{align*}
		f(n) \le c_2g(n) &\Rightarrow g(n) \ge \frac{1}{c_2}f(n) \\
		f(n) \ge c_1g(n) &\Rightarrow g(n) \le \frac{1}{c_1}f(n)
	\end{align*}

	بنابراین

	\begin{equation*}
		\frac{1}{c_2}f(n) \le g(n) \le \frac{1}{c_1}f(n) \; (\forall n \ge n_0)
	\end{equation*}

	پس
	$g(n) \in \Theta(n)$.
	اثبات طرف دیگر نیز کاملا مشابه اثبات بالا است.

	\subsection*{ج)}
	\paragraph*{}
	به ازای
	$f(n) = \frac{1}{n}$
	جمله نادرست می‌شود. زیرا اگر
	$\frac{1}{n} \in O(\frac{1}{n^2})$
	آن‌گاه به ازای یک
	$c_1, n_0$
	باید داشته باشیم

	\begin{equation*}
		\frac{1}{n} \le \frac{c_1}{n^2}
	\end{equation*}

	بنابراین

	\begin{align*}
		n &\ge \frac{1}{c_1}n^2 \\
		\Rightarrow
		1 &\ge \frac{n}{c_1} \\
		\Rightarrow
		n &\le c_1
	\end{align*}

	بنابراین نمی‌توان
	$n_0$ای
	پیدا کرد که به ازای هر
	$n \ge n_0$
	نامساوی بالا برقرار باشد. پس
	$\frac{1}{n} \notin O(\frac{1}{n^2})$
	و جمله در حالت کلی درست نیست.

	\subsection*{د)}
	\paragraph*{}

	به ازای
	$f(n) = 2n$
	و
	$g(n) = n$
	جمله نادرست می‌شود. مشابه سوال قبل، اگر
	$3^{2n} \in O(3^n)$
	آن‌گاه باید داشته باشیم

	\begin{align*}
		3^{2n} &\le c_13^n \\
		\Rightarrow
		3^n &\le c_1 \\
		\Rightarrow
		n &\le \log_3 c_1
	\end{align*}

	بنابراین نمی‌توان
	$n_0$ای
	پیدا کرد که به ازای هر
	$n \ge n_0$
	نامساوی بالا برقرار باشد. پس نتیجه می‌گیریم
	$3^{2n} \notin O(3^n)$
	و جمله در حالت کلی درست نیست.

	\section*{سوال ۲}
	\subsection*{الف)}
	\paragraph*{}

	فرض می‌کنیم $c$ای وجود دارد که
	\begin{equation*}
		[\lg n]! \in O(n^c)
	\end{equation*}
	در این صورت باید
	$k$
	و
	$n_0$ای
	وجود داشته باشد که
	\begin{equation*}
		[\lg n]! \le kn^c \; (\forall n > n_0)
	\end{equation*}

	و بنابراین
	\begin{equation*}
		\lg([\lg n]!) \le ck \lg n
	\end{equation*}

	بنا به تقریب استرلینگ می‌توان نشان داد
	\begin{align*}
		\lg(n!) &\approx \lg\left(\sqrt{2\pi n}\left(\frac{n}{e}\right)^n\right) \\
		&=
		\lg\left(\sqrt{2\pi n}\right)
		+
		\lg \left(\frac{n}{e}\right)^n \\
		&=
		\lg \sqrt{2\pi} + \lg \sqrt{n} + n\lg \left(\frac{n}{e}\right) \\
		&=
		\lg \sqrt{2\pi} + \frac{1}{2}\lg n + n\lg n - n\lg e \\
		&\in \Theta(n \lg n)
	\end{align*}

	به طور مشابه می‌توان نتیجه گرفت
	\begin{equation*}
		\lg([\lg n]!) \in \Theta([\lg n] \lg [\lg n])
	\end{equation*}

	چون به ازای
	$n \ge 2$،
	$\frac{1}{2}n \le [n] \le n$
	بنابراین
	$[n] \in \Theta(n)$
	و می‌توان از علامت کف صرف نظر کرد:
	\begin{equation*}
		\lg([\lg n]!) \in \Theta(\lg n \lg \lg n)
	\end{equation*}

	چون به ازای
	$n > 4$
	داریم
	$\lg n \lg \lg n > \lg n$
	بنابراین

	\begin{equation*}
		\lg([\lg n]!) \in \omega(\lg n)
	\end{equation*}

	و به عبارتی، پیچیدگی زمانی بالاتر از $\lg n$ داریم. پس نامساوی اولیه نمی‌تواند به ازای هیچ
	$c, k, n_0$ای
	درست باشد.

	\subsection*{ب)}
	\paragraph*{}

	مشابه قسمت قبل
	\begin{align*}
		\lg ([\lg \lg n]!) &\in \Theta \lg ([\lg \lg n] \lg [\lg \lg n]) \\
		&\in \Theta(\lg \lg n \lg \lg \lg n) \\
		&\in o(\lg \lg n \lg \lg n) \\
		&\in o(\lg^2 \lg n)
	\end{align*}

	لگاریتم (توانی) یک چیز، از خود آن چیز کندتر رشد می‌کند. پس
	\begin{equation*}
		\lg ([\lg \lg n]!) \in o(\lg^2 \lg n) \in O(\lg n)
	\end{equation*}

	پس می‌توان
	$c, k, n_0$ای
	پیدا کرد که در نامساوی قسمت قبل برقرار باشد. بنابراین
	$[\lg \lg n]! \in O(n^c)$.

	\section*{سوال ۳}
	\paragraph*{}

	برای سادگی فرض می‌کنیم
	\begin{gather*}
		T(1) = 0 \\
		f(n) = n^{\log_b a} \lg_k n
	\end{gather*}
	با جایگزینی داریم

	\begin{align*}
		T(n) &= aT(\frac{n}{b}) + f(n) \\
		&= a(aT(\frac{n}{b^2}) + f(\frac{n}{b})) + f(n) \\
		&= a^2 T(\frac{n}{b^2}) + a^1f(\frac{n}{b^1}) + f(n) \\
		&= ... \\
		&= \sum_{i=0}^{\log_b n - 1} a^i f(\frac{n}{b^i}) \\
		&= \sum_{i=0}^{\log_b n - 1} a^i \left(\frac{n}{b^i}\right)^{\log_b a} \left(\lg \left(\frac{n}{b^i}\right)\right)^k \\
		&= \sum_{i=0}^{\log_b n - 1} a^i \left(\frac{n^{\log_b a}}{b^{i{\log_b a}}}\right) \left(\lg \left(\frac{n}{b^i}\right)\right)^k \\
		&= \sum_{i=0}^{\log_b n - 1} a^i \left(\frac{n^{\log_b a}}{a^i}\right) \left(\lg \left(\frac{n}{b^i}\right)\right)^k \\
		&= \sum_{i=0}^{\log_b n - 1}  n^{\log_b a} \left(\lg \left(\frac{n}{b^i}\right)\right)^k \\
		&= \sum_{i=0}^{\log_b n - 1}  n^{\log_b a} \left(\frac{\log_b \left(\frac{n}{b^i}\right)}{\log_b 2}\right)^k \\
		&= n^{\log_b a} \log_b^{-k} 2 \sum_{i=0}^{\log_b n - 1} \left(\log_b n - i\right)^k \\
		&= n^{\log_b a} \log_b^{-k} 2 \sum_{j=1}^{\log_b n} j^k \\
	\end{align*}

		می‌توان مجموع را با انتگرال تقریب زد.

	\begin{align*}
		\sum_{j=1}^{\log_b n} j^k
		&\approx \int_{0}^{\log_b n} x^k dx\\
		&= \left.\frac{x^{k+1}}{k+1}\right|^{\log_b n}_{0} \\
		&= \frac{\log^{k+1}_b n}{k+1}
	\end{align*}

	درنهایت
	\begin{equation*}
		T(n) \in \Theta(n^{\log_b a} \log_b^{k+1} n)
	\end{equation*}

	در نتیجه برای
	\begin{equation*}
		T(n) = 4T(\frac{n}{2}) + n^2 \lg^5 n
	\end{equation*}

	به دست می‌آید
	\begin{equation*}
		T(n) \in \Theta(n^2 \lg^6 n)
	\end{equation*}

	\section*{سوال ۴}
	\paragraph*{}

	با توجه به این که
	\begin{enumerate}
		\item
		این تابع بازگشتی است؛

		\item
		شرط خاتمه آن فقط به مقدار
		\LRE{\verb|m|}
		بستگی دارد؛
	\end{enumerate}

	می‌توان نتیجه گرفت که تابع هزینه
	\LRE{\verb|function|}
	فقط تابعی از
	\LRE{\verb|m|}
	است و به
	\LRE{\verb|n|}
	بستگی ندارد.

	\paragraph*{}
	همچنین در این تابع، خود تابع صدا زده می‌شود و عملیات دیگر همه پیچیدگی زمانی
	$O(1)$
	دارند. بنابراین تابع هزینه
	\LRE{\verb|function|}
	را می‌توان به صورت
	\begin{equation*}
		T(m) = T(\frac{m}{2}) + c
	\end{equation*}

	نوشت (که در آن $c$ یک ثابت و نشان‌دهندهٔ پیچیدگی زمانی عملیات دیگر است). با استفاده از قضیه‌ای که در سوال ۳ اثبات کردیم، با مقایسه
	$m^{\log_2 1} \in \Theta(1)$
	و
	$c \in \Theta(1)$،
	نتیجه می‌گیریم
	$T(m) \in \Theta(\log m)$.

	\section*{سوال ۵}
	\subsection*{الف)}
	\paragraph*{}
	با رسم درخت بازگشت، می‌توان دید که
	مجموع هزینه ردیف $i$، برابر
	$\left(\frac{5}{8}\right)^in$
	است. همچنین ارتفاع درخت حداکثر برابر
	$[\log_4 n]$
	است. داریم

	\begin{align*}
		T(n) &\le \sum_{i=0}^{[\log_4 n]} \left(\frac{5}{8}\right)^in \\
		&\le n \sum_{i=0}^{\infty} \left(\frac{5}{8}\right)^i \\
		&\le n \left(\frac{1}{1-\frac{5}{8}}\right) \\
	\end{align*}

	در نتیجه
	\begin{equation*}
		T(n) \in O(n)
	\end{equation*}

	\subsection*{ب)}
	\paragraph*{}

	داریم
	\begin{equation*}
		T(n) - 4T(n-1) + 5T(n-2) - 2T(n-3) = 0
	\end{equation*}

	این رابطه بازگشتی همگن خطی است. معادله مشخصه را به صورت زیر نوشته و حل می‌کنیم.

	\begin{align*}
		x^3 - 4x^2 + 5x - 2 &= 0 \\
		(x-1)(x^2 - 3x + 2) &= 0 \\
		(x-1)(x-1)(x-2) &= 0 \\
		\Rightarrow
		\begin{cases}
			x = 1, r_1 = 2 \\
			x = 2, r_2 = 1
		\end{cases}
	\end{align*}

	جواب عمومی به صورت زیر خواهد بود.

	\begin{equation*}
		T(n) = c_1r1^n + c_2n1^n + c_32^n
	\end{equation*}

	با فرض مثبت بودن همه ضرایب،
	$T(n) \in \Theta(2^n)$.

	\subsection*{ج)}
	\paragraph*{}

	تغییر متغیر
	$S(n) = \frac{T(n)}{n}$
	را انجام می‌دهیم. داریم

	\begin{align*}
		S(n) &= \frac{T(n)}{n} = 4\frac{T(\sqrt{n})}{\sqrt{n}} + \lg^2 n \lg^3 \lg n \\
		&= 4S(\sqrt{n}) + \lg^2 n \lg^3 \lg n
	\end{align*}

	دوباره تغییر متغیر
	$P(i) = S(2^i)$
	را انجام می‌دهیم. داریم

	\begin{align*}
		P(i) &= S(2^i) = 4S(2^{\frac{i}{2}}) + i^2 \lg^3 i \\
		&= 4P(\frac{i}{2}) + i^2 \lg^3 i \\
	\end{align*}

	با توجه به قضیه‌ای که در سوال ۳ ثابت کردیم، داریم
	\begin{equation*}
		P(i) \in \Theta(i^2 \lg^4 i) \\
	\end{equation*}

	چون
	$n = 2^i$،
	بنابراین
	$i = \lg n$
	و
	\begin{equation*}
		S(n) = P(\lg n) \in \Theta(\lg^2 n \lg^4 \lg n)
	\end{equation*}

	همچنین
	\begin{equation*}
		T(n) = nS(n) \in \Theta(n \lg^2 n \lg^4 \lg n)
	\end{equation*}


\end{document}