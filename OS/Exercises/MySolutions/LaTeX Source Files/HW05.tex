\documentclass{article}

%\usepackage{graphicx}
%\usepackage{float}
\usepackage{listings}
\usepackage{lstautogobble}
\usepackage[letterpaper, margin=1in]{geometry}
%\usepackage{enumitem}
\usepackage{amsmath}
%\usepackage[normalem]{ulem}
\usepackage{pgfgantt}
\usepackage{xepersian}

%\graphicspath{{./img}}
\settextfont{XB Niloofar}

\title{سیستم‌های عامل - دکتر ابراهیمی‌مقدم}
\author{امیرحسین منصوری - ۹۹۲۴۳۰۶۹}
\date{تمرین سری پنجم}

\begin{document}
	\maketitle

	\section*{سوال ۱}
	\subsection*{الف)}
	\paragraph*{نادرست؛}
	در یک سیستم با
	$n$
	پروسه، بدترین زمان پاسخگویی در
	\LRE{RR}
	برای هر پروسه، برابر
	$(n-1)q$
	است. در یک سناریوی فرضی، اگر
	$n=3$
	 پروسه داشته باشیم که هر کدام بخواهند ۳ ثانیه اجرا شوند، در
	\LRE{SJF}
	زمان پاسخگویی برابر
	$\frac{0 + 3s + 6s}{3} = 3s$
	خواهد بود. در حالی که اگر از
	\LRE{RR}
	با
	$q=10ms$
	استفاده کنیم، بیشترین میانگین زمان ممکن برابر
	$(n-1)q = 20ms$
	خواهد بود که بسیار کمتر از معادل آن در
	\LRE{SJF}
	است.

	\subsection*{ب)}
	\paragraph*{درست...؛}
	از نظر متوسط زمان پاسخگویی پروسه‌ها، این جمله درست است. اما معمولا
	\LRE{FIFO}
	سربار کمتری نسبت به
	\LRE{SJF}
	دارد و از این لحاظ کمی سریع‌تر است.

	\subsection*{ج)}
	\paragraph*{نادرست؛}
	در این حالت، زمان‌بند برخی پروسه‌ها را از حافظه اصلی خارج و در
	\LRE{Virtual memory}
	ذخیره می‌کند (اصطلاحاً
	\LRE{Swap out}
	می‌کند) تا جا برای پروسه‌های دیگر در حافظه اصلی باز شود.

	\section*{سوال ۲}
	\subsection*{\LRE{SJF}}

	\begin{latin}
		\resizebox{\textwidth}{!}{
		\begin{ganttchart}[
			expand chart=\textwidth,
			vgrid,
			hgrid,
			inline
			]{0}{12}
			\gantttitlelist{0,...,12}{1} \\
			\ganttbar{P2}{0}{3} \\
			\ganttbar{P3}{4}{4} \\
			\ganttbar{P1}{5}{12}
			\ganttvrule{P1 and P2 Enter}{-1}
			\ganttvrule{P3 Enters}{1}
		\end{ganttchart}
		}
	\end{latin}

	\paragraph*{}
	در نمودار بالا، ابتدا
	\LRE{P1}
	و
	\LRE{P2}
	با هم وارد می‌شوند. چون
	\LRE{Burst time}
	پروسهٔ
	\LRE{P2}
	کمتر است، این پروسه برای اجرا انتخاب می‌شود. پس از اتمام اجرا،
	\LRE{P3}
	و
	\LRE{P1}
	در سیستم هستند و چون
	\LRE{P3}
	زمان
	\LRE{Burst time}
	کمتری دارد، برای اجرا انتخاب می‌شود. در نهایت
	\LRE{P1}
	نیز اجرا می‌شود.

	\paragraph*{}
	با توجه به شکل:

	\begin{gather*}
		\textrm{\LRE{Response Time for P1}} = 5 \\
		\textrm{\LRE{Response Time for P2}} = 0 \\
		\textrm{\LRE{Response Time for P3}} = 4 \\
		\Rightarrow
		\textrm{\LRE{Average Response Time}} =
		\frac{5 + 0 + 4}{3} = 3
	\end{gather*}

	\subsection*{\LRE{FIFO} (یا \LRE{FCFS})}

	\begin{latin}
		\resizebox{\textwidth}{!}{
			\begin{ganttchart}[
				expand chart=\textwidth,
				vgrid,
				hgrid,
				inline
				]{0}{12}
				\gantttitlelist{0,...,12}{1} \\
				\ganttbar{P1}{0}{7} \\
				\ganttbar{P2}{8}{11} \\
				\ganttbar{P3}{12}{12}
				\ganttvrule{P1 and P2 Enter}{-1}
				\ganttvrule{P3 Enters}{1}
			\end{ganttchart}
		}
	\end{latin}

	\paragraph*{}
	بین پروسه‌های
	\LRE{P1}
	و
	\LRE{P2}
	که همزمان وارد می‌شوند، ابتدا
	\LRE{P1}
	که اولویت بیشتری دارد اجرا می‌شود و سپس
	\LRE{P2}
	اجرا می‌شود. در نهایت
	\LRE{P3}
	که از همه دیرتر وارد شده، اجرا می‌شود. داریم:

	\begin{gather*}
		\textrm{\LRE{Response Time for P1}} = 0 \\
		\textrm{\LRE{Response Time for P2}} = 8 \\
		\textrm{\LRE{Response Time for P3}} = 12 \\
		\Rightarrow
		\textrm{\LRE{Average Response Time}} =
		\frac{0 + 8 + 12}{3} = \frac{20}{3} \approx 6.67
	\end{gather*}

	\subsection*{\LRE{SRTF}}

	\begin{latin}
		\resizebox{\textwidth}{!}{
			\begin{ganttchart}[
				expand chart=\textwidth,
				vgrid,
				hgrid,
				inline
				]{0}{12}
				\gantttitlelist{0,...,12}{1} \\
				\ganttbar{P2}{0}{1} \\
				\ganttbar{P3}{2}{2} \\
				\ganttbar{P2}{3}{4} \\
				\ganttbar{P1}{5}{12}
				\ganttvrule{P1 and P2 Enter}{-1}
				\ganttvrule{P3 Enters}{1}
			\end{ganttchart}
		}
	\end{latin}

	\paragraph*{}
	از بین پروسه‌های
	\LRE{P1}
	و
	\LRE{P2}،
	پروسهٔ
	\LRE{P2}
	که زمان باقی‌مانده‌ٔ کوتاه‌تری دارد (برابر ۴)، وارد می‌شود. در زمان
	$t + 2$،
	پروسهٔ
	\LRE{P3}
	وارد می‌شود و چون کوتاه‌ترین زمان باقی‌مانده را دارد (برابر ۱)، جایگزین پروسهٔ قبلی می‌شود و اجرا می‌شود. بعد از پایان این پروسه، پروسهٔ
	\LRE{P2}
	که ۲ واحد زمانی نیاز دارد تا به اتمام برسد (در مقابل
	\LRE{P1}
	که ۸ واحد زمانی نیاز دارد)، به اجرا شدن ادامه می‌دهد. بعد از اتمام این پروسه،
	\LRE{P1}
	در نهایت اجرا می‌شود. داریم:

	\begin{gather*}
		\textrm{\LRE{Response Time for P1}} = 5 \\
		\textrm{\LRE{Response Time for P2}} = 0 \\
		\textrm{\LRE{Response Time for P3}} = 2 \\
		\Rightarrow
		\textrm{\LRE{Average Response Time}} =
		\frac{5 + 0 + 2}{3} = \frac{7}{3} \approx 2.33
	\end{gather*}

	\section*{سوال ۳}

	\subsection*{الف)}
	\paragraph*{}
	همهٔ پروسه‌های وارد شده در سیستم، فارغ از وضعیتشان، اولویت خود را به مرور از دست می‌دهند. اما چون
	$b < a$،
	پروسه‌های در وضعیت
	\LRE{Running}
	اولویت خود را سریع‌تر از پروسه‌های
	\LRE{Ready}
	از دست می‌دهند. این باعث می‌شود که یک پروسهٔ
	\LRE{Running}،
	بعد از مدتی جای خود را با یک پروسهٔ
	\LRE{Ready}
	عوض کند. همچنین چون اولویت یک پروسهٔ جدید برابر صفر است، از همه پروسه‌های موجود اولویت بیشتری خواهد داشت و بنابراین به هنگام ورود، بلافاصله اجرا می‌شود. در دو حالت می‌توان الگوریتم حاصل را بررسی کرد:

	\begin{itemize}
		\item{\textbf{همه پروسه‌ها با هم وارد شوند:}}
		در این حالت، الگوریتم دقیقا مثل
		\LRE{RR}
		عمل خواهد کرد که
		\LRE{Time quantum}
		کوچکی دارد.

		\item{\textbf{پروسه‌ها در زمان‌های مختلف وارد شوند:}}
		در این حالت، اولویت با پروسه‌ای است که کمترین
		\LRE{CPU Time}
		تا به حال به آن اختصاص داده شده؛ مثلا پروسهٔ جدیدی که تا به حال اجرا نشده، بیشترین اولویت را پیدا می‌کند و بلافاصله شروع به اجرا می‌کند. اگر برای مدت طولانی پروسهٔ جدیدی وارد نشود، بعد از مدتی اولویت پروسه‌ها نزدیک به هم می‌شود و الگوریتم دوباره مثل
		\LRE{RR}
		عمل می‌کند.
	\end{itemize}

	\subsection*{ب)}
	\paragraph*{}
	چون همواره اولویت پروسه‌های موجود در حال کم شدن است، پروسه‌ای که تازه وارد می‌شود، همیشه بیشترین اولویت را خواهد داشت و بلافاصله جایگزین پروسهٔ فعلی در حال اجرا می‌شود. همچنین پروسه‌های در وضعیت
	\LRE{Running}
	کندتر از پروسه‌های در وضعیت
	\LRE{Ready}
	اولویت خود را از دست می‌دهند. بنابراین یک پروسهٔ در وضعیت
	\LRE{Running}،
	همیشه از همهٔ پروسه‌های
	\LRE{Ready}
	موجود اولویت بیشتری خواهد داشت و بنابراین تا اتمام پروسه، در وضعیت
	\LRE{Running}
	باقی خواهد ماند. همچنین چون پروسه‌ها در ابتدا اولویت صفر دارند، هرچه کمتر در صف
	\LRE{Ready}
	بوده باشند، یا به عبارتی
	\textit{دیرتر}
	وارد شده باشند، اولویت بیشتری دارند.

	\paragraph*{}
	در نهایت می‌توان این یک الگوریتم
	\LRE{Pre-emptive LIFO}
	است؛ که یعنی بیشترین اولویت را به پروسه‌های جدید می‌دهد
	(\LRE{Last In, First Out})
	و در صورت وارد شدن یک پروسه جدید، بلافاصله پردازنده را به آن اختصاص می‌دهد
	(\LRE{Pre-emptive}).

	\subsection*{ج)}
	\paragraph*{}
	چون
	$b > 0$،
	اولویت پروسه‌های در وضعیت
	\LRE{Running}
	همواره در حال افزایش، و چون
	$a < 0$،
	اولویت پروسه‌های در وضعیت
	\LRE{Ready}
	همواره در حال کاهش است. بنابراین یک پروسهٔ در وضعیت
	\LRE{Running}،
	از هر پروسهٔ دیگر (چه جدید باشد، چه
	\LRE{Ready})
	اولویت بیشتری دارد و بنابراین هیچ‌گاه اجرای آن متوقف نمی‌شود (دچار
	\LRE{Pre-emption}
	نمی‌شویم).
	همچنین مانند قسمت قبل، بین پروسه‌های
	\LRE{Ready}،
	آن که کمتر در صف بوده باشد و به عبارتی جدیدتر آمده باشد، اولویت بیشتری دارد.

	\paragraph*{}
	در نهایت می‌توان گفت این یک الگوریتم
	\LRE{Non-preemptive LIFO}
	است؛ چون مانند قسمت قبل، بیشترین اولویت را به پروسه‌های جدید‌تر می‌دهد؛ اما در عمل هیچ‌گاه
	\LRE{Pre-emption}
	در آن رخ نمی‌دهد و یک پروسهٔ در حال اجرا، تا اتمام کارش پردازنده را در اختیار خواهد داشت.

	\section*{سوال ۴}
	\paragraph*{}
	نمودار
	\LRE{Gantt chart}
	پروسه‌ها به صورت زیر است. به علت طولانی بودن محاسبات عددی مربوط به آن، از آوردن این محاسبات صرف‌نظر کرده‌ایم!

	\paragraph*{}

	\begin{latin}
		\begin{ganttchart}[
			expand chart=\textwidth,
			%vgrid,
			hgrid,
			inline
			]{0}{19}
			%\gantttitlelist{0,2,...,21}{2}\\
			\ganttbar{P1}{0}{0}
			\ganttbar{P1/P2}{1}{3}
			\ganttbar{P1/P2/P3}{4}{6}
			\ganttbar{P2/P3}{7}{8}
			\ganttbar{P2/P3/P4}{9}{11}
			\ganttbar{P2/P3/P4/P5}{12}{15}
			\ganttbar{P2/P3/P4/P5/P6}{16}{19}
			\ganttvrule{0}{-1}
			\ganttvrule{1}{0}
			\ganttvrule{4}{3}
			\ganttvrule{5.5}{6}
			\ganttvrule{7}{8}
			\ganttvrule{9}{11}
			\ganttvrule{12}{15}
			\ganttvrule{13.25}{19}
		\end{ganttchart}

		\paragraph*{}

		\begin{ganttchart}[
			expand chart=0.99\textwidth,
			%vgrid,
			hgrid,
			inline
			]{0}{16}
			%\gantttitlelist{0,2,...,21}{2}\\
			\ganttbar{P2/P3/P4/P6}{0}{2}
			\ganttbar{P2/P4/P6}{3}{6}
			\ganttbar{P4/P6}{7}{10}
			\ganttbar{P4}{11}{16}
			\ganttvrule{13.25}{-1}
			\ganttvrule{13.58}{2}
			\ganttvrule{15.08}{6}
			\ganttvrule{17.42}{10}
			\ganttvrule{21}{16}
		\end{ganttchart}

	\end{latin}

	\paragraph*{}
	برای به دست آوردن نمودار بالا، کافی است محاسبه کنیم در زمان وارد شدن هر پروسه، چند پروسه همزمان داریم؛ و این پروسه‌های همزمان تا چه زمانی با هم اجرا می‌شوند، یا کدام‌یک از آن‌ها در چه زمانی به اتمام می‌رسند. به طور کلی اگر $n$ پروسه همزمان در سیستم داشته باشیم، و کمترین زمان باقی‌مانده بین این
	$n$
	پروسه برابر
	$t$
	باشد، این پروسه‌ها تا
	$n \times t$
	زمان بعد با هم اجرا می‌شوند؛ و درست بعد از این زمان پروسه با کمترین زمان باقی‌مانده به اتمام می‌رسد و
	$n-1$
	پروسه باقی می‌مانند. همچنین اگر
	$n$
	پروسه با هم در سیستم باشند، و زمان
	$q$
	بگذرد، از زمان باقی‌مانده هر کدام از پروسه‌ها
	$\frac{q}{n}$
	کم می‌شود. با استفاده از این روابط، اعداد نمودار بالا قابل محاسبه است.

	\paragraph*{}
	زمان انتظار هر پروسه نیز برابر
	\begin{equation*}
		(\textrm{\LRE{Finish time}} - \textrm{\LRE{Arrival time}}) - \textrm{\LRE{Burst time}}
	\end{equation*}
	خواهد بود. بنابراین:

	\begin{align*}
		W_{P_1} &= (5.5 - 0) - 3 = 2.5 \\
		W_{P_2} &= (15.08 - 1) - 5 = 9.08 \\
		W_{P_3} &= (13.58 - 4) - 3 = 6.58 \\
		W_{P_4} &= (21 - 7) - 7 = 7 \\
		W_{P_5} &= (13.25 - 9) - 1 = 3.25 \\
		W_{P_6} &= (17.42 - 12) - 2 = 3.42
	\end{align*}

	و میانگین این زمان‌ها برابر با
	۵٫۳۰۵
	دقیقه خواهد بود.

	\section*{سوال ۵}
	\subsection*{الف)}
	\paragraph*{}
	برای این که مدیریت این پروسه‌ها ممکن باشد، باید داشته باشیم
	$\sum_{i=1}^{m} \frac{c_i}{p_i} \le 1$.
	بنابراین

	\begin{align*}
		\frac{x}{100} + \frac{40}{200} + \frac{100}{500} &\le 1 \\
		\Rightarrow
		x &\le 60
	\end{align*}

	\subsection*{ب)}
	برای این که مدیریت این پروسه‌ها با الگوریتم
	\LRE{RMS}
	ممکن باشد، باید داشته باشیم
	$\sum_{i=1}^{m} \frac{c_i}{p_i} \le m(2^{\frac{1}{m}} - 1)$.
	بنابراین

	\begin{align*}
		\frac{x}{100} + \frac{40}{200} + \frac{100}{500} &\le 3(2^{\frac{1}{3}} - 1) \approx 0.78 \\
		\Rightarrow
		x \le 37.97
	\end{align*}

	\subsection*{ج)}

	در هر دو مثال، فرض می‌کنیم
	$x = 30$.

	\subsubsection*{\LRE{RMS}}

	\begin{latin}
		\begin{ganttchart}[
			expand chart=\textwidth,
			vgrid,
			%hgrid,
			inline
			]{0}{99}
			\gantttitlelist{0,100,...,999}{10} \\

			\ganttbar{P1}{0}{2}
			\ganttbar{P2}{3}{6}
			\ganttbar{P3}{7}{9}

			\ganttbar{P1}{10}{12}
			\ganttbar{P3}{13}{19}
			%\ganttbar{P3}{7}{9}

			\ganttbar{P1}{20}{22}
			\ganttbar{P2}{23}{26}

			\ganttbar{P1}{30}{32}

			\ganttbar{P1}{40}{42}
			\ganttbar{P2}{43}{46}

			\ganttbar{P1}{50}{52}
			\ganttbar{P3}{53}{59}

			\ganttbar{P1}{60}{62}
			\ganttbar{P2}{63}{66}
			\ganttbar{P3}{67}{69}

			\ganttbar{P1}{70}{72}

			\ganttbar{P1}{80}{82}
			\ganttbar{P2}{83}{86}

			\ganttbar{P1}{90}{92}

			\ganttvrule{$T_1$}{9}
			\ganttvrule{$T_1$, $T_2$}{19}
			\ganttvrule{$T_1$}{29}
			\ganttvrule{$T_1$, $T_2$}{39}
			\ganttvrule{$T_1$, $T_3$}{49}
			\ganttvrule{$T_1$, $T_2$}{59}
			\ganttvrule{$T_1$}{69}
			\ganttvrule{$T_1$, $T_2$}{79}
			\ganttvrule{$T_1$}{89}
			\ganttvrule{$T_1$, $T_2$}{99}

			%\ganttvrule{T1}{9}

		\end{ganttchart}
	\end{latin}

	\subsubsection*{\LRE{EDF}}

	\begin{latin}
		\begin{ganttchart}[
			expand chart=\textwidth,
			vgrid,
			%hgrid,
			inline
			]{0}{99}
			\gantttitlelist{0,100,...,999}{10} \\

			\ganttbar{P1}{0}{2}
			\ganttbar{P2}{3}{6}
			\ganttbar{P3}{7}{9}

			\ganttbar{P1}{10}{12}
			\ganttbar{P3}{13}{19}
			%\ganttbar{P3}{7}{9}

			\ganttbar{P1}{20}{22}
			\ganttbar{P2}{23}{26}

			\ganttbar{P1}{30}{32}

			\ganttbar{P1}{40}{42}
			\ganttbar{P2}{43}{46}

			\ganttbar{P1}{50}{52}
			\ganttbar{P3}{53}{59}

			\ganttbar{P1}{60}{62}
			\ganttbar{P2}{63}{66}
			\ganttbar{P3}{67}{69}

			\ganttbar{P1}{70}{72}

			\ganttbar{P1}{80}{82}
			\ganttbar{P2}{83}{86}

			\ganttbar{P1}{90}{92}

			\ganttvrule{$T_1$}{9}
			\ganttvrule{$T_1$, $T_2$}{19}
			\ganttvrule{$T_1$}{29}
			\ganttvrule{$T_1$, $T_2$}{39}
			\ganttvrule{$T_1$, $T_3$}{49}
			\ganttvrule{$T_1$, $T_2$}{59}
			\ganttvrule{$T_1$}{69}
			\ganttvrule{$T_1$, $T_2$}{79}
			\ganttvrule{$T_1$}{89}
			\ganttvrule{$T_1$, $T_2$}{99}

			%\ganttvrule{T1}{9}

		\end{ganttchart}
	\end{latin}

	\paragraph*{}
	نمودار مربوط به الگوریتم
	\LRE{EDF}،
	دقیقا مانند
	\LRE{RMS}
	به دست می‌آید.

\end{document}