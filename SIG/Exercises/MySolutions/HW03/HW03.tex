\documentclass{article}

\usepackage{amsmath}
\usepackage{xepersian}

\settextfont{XB Niloofar}

\title{سیگنال‌ها و سیستم‌ها - دکتر سلیمی بدر}
\author{امیرحسین منصوری - ۹۹۲۴۳۰۶۹}
\date{تمرین سری ۳}

\begin{document}
	\maketitle

	\section*{سوال ۱}
	\paragraph*{}

	پاسخ سیستم برابر کانوولوشن سیگنال
	$x[n]$
	و
	$h[n]$
	است.

	\begin{align*}
		y[n] &= x[n] * h[n]
		= \sum_{k=-\infty}^{\infty} x[k]h[n-k] \\
		&= \sum_{k=1}^{3} 2 h[n-k] = 2 \times \sum_{k=1}^{3} h[n-k]
	\end{align*}

	حال
	$y[n]$
	را بازه‌بندی می‌کنیم.
	\begin{align*}
		n \le 2 &\Rightarrow y[n] = 0 \\
		3 \le n \le 4 &\Rightarrow y[n] = 2 \times \sum_{k=1}^{n-2} 1 = 2(n-2) \\
		5 \le n \le 7 &\Rightarrow y[n] = 2 \times \sum_{k=1}^{3} 1 = 6 \\
		8 \le n \le 9 &\Rightarrow y[n] = 2 \times \sum_{k=n-6}^{3} 1 = 2(10-n) \\
		n > 9 &\Rightarrow y[n] = 0
	\end{align*}

	مشخص است که در بازه
	$5 \le n \le 7$،
	سیگنال بیشترین مقدار را دارد.

	\section*{سوال ۲}
	\subsection*{الف)}
	\paragraph*{}

	\begin{align*}
		y(t) &= \int_{-\infty}^{\infty} x(\tau) h(t-\tau) d \tau \\
		&= \int_{-\infty}^{\infty} e^{-|\tau|} e^{-2(t-\tau+1)} u(t-\tau+1) d \tau \\
		&= \int_{-\infty}^{1+t} e^{-|\tau|} e^{-2(t-\tau+1)} d \tau \\
		&= e^{-2t-2} \int_{-\infty}^{1+t} e^{-|\tau|} e^{2\tau} d \tau \\
	\end{align*}

	حال
	$y(t)$
	را بازه‌بندی می‌کنیم.
	\begin{align*}
		t < -1 \Rightarrow y(t) &= e^{-2t-2} \int_{-\infty}^{1+t} e^{-(-\tau)} e^{2\tau} d \tau \\
		&= e^{-2t-2} \int_{-\infty}^{1+t} e^{3\tau} d \tau \\
		&= \frac{1}{3} e^{-2t-2} (e^{t+1}) \\
		&= \frac{1}{3} e^{-t-1} \\
		t \ge -1 \Rightarrow y(t) &= e^{-2t-2} \int_{-\infty}^{1+t} e^{-\tau} e^{2\tau} d \tau \\
		&= e^{-2t-2} \int_{-\infty}^{1+t} e^{\tau} d \tau \\
		&= e^{-2t-2} e^{t+1} \\
		&= e^{-t-1}
	\end{align*}

	\subsection*{ب)}
	\paragraph*{}

	\begin{align*}
		y(t) &= \int_{-\infty}^{\infty} x(\tau) h(t-\tau) d \tau \\
		&= \int_{-\infty}^{\infty} 2e^{-3 \tau} u(\tau) (e^{-t+\tau} - e^{-2t + 2\tau}) u(t - \tau) d\tau \\
		&= \int_{0}^{t} 2e^{-3 \tau} (e^{-t+\tau} - e^{-2t + 2\tau}) d\tau \\
		&= 2 \int_{0}^{t} e^{-t-2\tau} - e^{-2t - \tau} d\tau \\
		&= 2(e^{-t} \int_{0}^{t} e^{-2\tau} d\tau + e^{-2t} \int_{0}^{t} e^{-\tau} d\tau) \\
		&= 2(-\frac{1}{2} e^{-t} (e^{-2t}-1) - e^{-2t} (e^{-t} - 1)) \\
		&= -e^{-3t} + e^{-t} - 2e^{-3t} + 2e^{-2t} \\
		&= -3e^{-3t} + 2e^{-2t} + e^{-t}
	\end{align*}

	\section*{سوال ۳}
	\subsection*{الف)}
	\paragraph*{}

	\begin{align*}
		y[n] &= \sum_{k=-\infty}^{\infty} \left(-\frac{1}{2}\right)^n u[k] u[n-k] \\
		&= \sum_{k=0}^{n} \left(-\frac{1}{2}\right)^n \\
		&= \frac{1 - \left(-\frac{1}{2}\right)^{n+1}}{\frac{3}{2}}
		= \frac{2}{3}(1 - \left(-\frac{1}{2}\right)^{n+1})
	\end{align*}

	\subsection*{ب)}
	\begin{align*}
		y[n] &= \sum_{k=-\infty}^{\infty} u[k]h[n-k] \\
		&= \sum_{k=0}^{\infty} (-1)^{n-k} (u[n-k+2] - u[n-k-3]) \\
		&= (-1)^n \sum_{k=0}^{\infty} (-1)^{k} (u[n-k+2] - u[n-k-3])
	\end{align*}

	حال
	$y[n]$
	را بازه‌بندی می‌کنیم.

	\begin{align*}
		n \le -3 \Rightarrow y[n] &= 0 \\
		-2 \le n \le 2 \Rightarrow
		y[n] &= (-1)^n \sum_{k=0}^{n+2} (-1)^{k} \\
		&= \begin{cases}
			\begin{matrix}
				1 & n \in E \\
				0 & n \in O
			\end{matrix}
		\end{cases} \\
		n \ge 3 \Rightarrow
		y[n] &= (-1)^n \sum_{k=n-2}^{n+2} (-1)^{k} \\
		&= \begin{cases}
			\begin{matrix}
				1 & n \in E \\
				1 & n \in O
			\end{matrix}
		\end{cases} \\
		&= 1
	\end{align*}

	\section*{سوال ۴}
	\paragraph*{}
	اگر
	$h[n]$
	را سیگنال پاسخ ضربه در نظر بگیریم، داریم

	\begin{align*}
		s[n] &= h[n] * u[n] \\
		&= \sum_{k = -\infty}^{+\infty} h[k]u[n-k] \\
		&= \sum_{k = -\infty}^{n} h[k] \\
		&= \sum_{k = -\infty}^{n-1} h[k] + h[n] \\
		&= s[n-1] + h[n] \\
		\Rightarrow
		h[n] &= s[n] - s[n-1] \\
		&= \left(\frac{1}{2}\right)^n u[n] - \left(\frac{1}{2}\right)^{n-1} u[n-1]
	\end{align*}

	\section*{سوال ۵}
	\subsection*{الف)}
	\paragraph{حافظه‌دار بودن}
	برای این که سیستم بدون حافظه باشد، باید
	$h[n]$
	در
	$n$
	های غیر صفر، مقدار صفر داشته باشد. اما
	$h[1] = \sin(\frac{\pi}{2}) = 1$
	و بنابراین سیستم حافظه‌دار است.

	\paragraph{علّی بودن}
	برای علّی بودن سیستم، برای
	$n$
	های منفی،
	$h[n]$
	باید صفر باشد. اما
	$h[-1] = \sin(-\frac{\pi}{2}) = -1$
	و بنابراین سیستم علّی نیست.

	\paragraph*{پایداری}
	برای پایداری، باید داشته باشیم
	$\sum_{k=-\infty}^{\infty} |h[k]| < L_2$.
	اما
	\begin{equation*}
		\sum_{k=-\infty}^{\infty} |\sin(\frac{\pi}{2}k)| = \infty
	\end{equation*}

	و چنین
	$L_2$
	ای وجود ندارد. بنابراین سیستم پایدار نیست.

	\subsection*{ب)}
	\paragraph{حافظه‌دار بودن}

	برای این که سیستم بدون حافظه باشد، باید
	$h[n]$
	در
	$n$
	های غیر صفر، مقدار صفر داشته باشد. اما
	$h[8] = \cos(\pi) = -1$.
	بنابراین سیستم حافظه‌دار است.

	\paragraph{علّی بودن}
	برای
	$n < 0$،
	داریم
	$u[n] - u[n-10] = 0$
	و در نتیجه
	$h[n] = 0$.
	در نتیجه سیستم علّی است.

	\paragraph*{پایداری}
	داریم
	\begin{align*}
		&\sum_{k=-\infty}^{\infty} |\cos(\frac{\pi}{8}k) (u[k] - u[k - 10])| \\
		&= \sum_{k=0}^{9} |\cos(\frac{\pi}{8}k)| < 10
	\end{align*}

	در نتیجه سیستم پایدار است.

	\section*{سوال ۶}
	\paragraph*{}

	\begin{align*}
		h[n] &= \sum_{k=0}^{n} e^{-2(n-k)} \delta[k-1] \\
		&= \begin{cases}
			\begin{matrix}
				0 & n < 0 \\
				e^{-2(n-1)} & n \ge 0
			\end{matrix}
		\end{cases}
	\end{align*}

	سیستم علّی است. زیرا در
	$n$
	های منفی، مقدار صفر دارد.

	\paragraph*{}
	همچنین برای پایداری سیستم داریم
	\begin{align*}
		\sum_{k=-\infty}^{\infty} |e^{-2(n-1)}| &= \sum_{k=0}^{\infty} e^{-2(n-1)} = e^2 + e^0 + e^{-2} + ... \\
		&= \frac{1}{1-\frac{1}{e^2}}
	\end{align*}

	در نتیجه سیستم پایدار است.

	\section*{سوال ۷}

	\begin{align*}
		\int_{-\infty}^{+\infty} z(\lambda) d\lambda
		&= \int_{-\infty}^{+\infty} \int_{-\infty}^{+\infty} x(\tau) y(\lambda - \tau) d\tau d\lambda \\
		&= \int_{-\infty}^{+\infty} \left(\int_{-\infty}^{+\infty} x(\tau) y(\lambda - \tau) d\lambda \right) d\tau \\
		&= \int_{-\infty}^{+\infty} \left(x(\tau) \int_{-\infty}^{+\infty} y(\lambda - \tau) d\lambda \right) d\tau \\
		&= \int_{-\infty}^{+\infty} \left(x(\tau) A_2 \right) d\tau \\
		&= A_2 \int_{-\infty}^{+\infty} x(\tau) d\tau \\
		&= A_1 A_2
	\end{align*}

	\section*{سوال ۸}
	\subsection*{الف)}
	\begin{align*}
		r_{xx}(t) &= x(t) * x(-t) = x(-t) * x(t) \\
		&= x(-t) * x(-(-t)) \\
		&= r_{xx}(-t)
	\end{align*}

	\subsection*{ب)}
	\begin{align*}
		r_{xy}(t) &= x(t) * y(-t) = y(-t) * x(t) \\
		&= y(-t) * x(-(-t)) \\
		&= r_{yx}(-t)
	\end{align*}

\end{document}