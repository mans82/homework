\documentclass{article}

\usepackage{amsmath}
\usepackage{xepersian}

\settextfont{XB Niloofar}

\title{سیگنال‌ها و سیستم‌ها - دکتر سلیمی بدر}
\author{امیرحسین منصوری - ۹۹۲۴۳۰۶۹}
\date{تمرین سری ۴}

\begin{document}
	\maketitle

	\section*{سوال ۱}
	\subsection*{الف)}
	\paragraph*{}
	داریم
	$\sin^2(4t) = \frac{1}{2} - \frac{1}{2}\cos(8t)$.
	در نتیجه داریم
	$\omega_0 = 8$.
	همچنین داریم
	\begin{equation*}
		\cos(8t) = \frac{e^{8jt} + e^{-8jt}}{2}
	\end{equation*}

	در نتیجه
	\begin{equation*}
		\sin^2(4t) = -\frac{1}{4} e^{-8jt} + \frac{1}{2} - \frac{1}{4} e^{8jt}
	\end{equation*}

	و ضرایب سری فوریه به سادگی مشخص می‌شود (بقیه ضرایب برابر صفر هستند).
	\begin{equation*}
		a_{-1} = -\frac{1}{4}, a_0 = \frac{1}{2}, a_1 = -\frac{1}{4}
	\end{equation*}

	\subsection*{ب)}
	\paragraph*{}

	برای این سیگنال می‌توان دوره تناوب
	$2\pi$
	را در نظر گرفت و در نتیجه
	$\omega_0 = \frac{2\pi}{2\pi} = 1$.
	داریم
	\begin{align*}
		x_2(t) &= \cos(3t) + \cos(5t) \\
		&= \frac{e^{3jt} + e^{-3jt}}{2} + \frac{e^{5jt} + e^{-5jt}}{2} \\
		&= \frac{1}{2} \left(e^{j(3)(1)t} + e^{j(-3)(1)t} + e^{j(5)(1)t} + e^{j(-5)(1)t}\right)
	\end{align*}

	و ضرایب سری فوریه به سادگی مشخص می‌شوند (بقیه ضرایب برابر صفر هستند).

	\begin{equation*}
		a_3 = a_{-3} = a_5 = a_{-5} = \frac{1}{2}
	\end{equation*}

	\subsection*{ج)}
	\paragraph*{}
	می‌توان دوره تناوب
	$T_0 = 2$
	را برای این سیگنال در نظر گرفت:

	\begin{equation*}
		x_3(t + 2) = 2\sin(3\pi t + 6\pi) + \cos(4\pi t + 8\pi)
		= 2\sin(3\pi t) + \cos(4\pi t) = x_3(t)
	\end{equation*}

	در نتیجه
	$\omega_0 = \frac{2\pi}{2} = \pi$.
	داریم

	\begin{align*}
		2\sin(3\pi t) &= \frac{e^{3\pi j t} - e^{- 3\pi j t}}{j} \\
		&= (-j)e^{j(3)(\pi)t} + je^{j(-3)(\pi)t}
	\end{align*}

	\begin{align*}
		\cos(4\pi t) &= \frac{e^{4\pi j t} + e^{- 4\pi j t}}{2} \\
		&= \frac{1}{2} e^{j(4)(\pi)t} + \frac{1}{2} e^{j(-4)(\pi)t}
	\end{align*}

	در نتیجه ضرایب سری فوریه مشخص می‌شوند (بقیه ضرایب برابر صفر هستند).

	\begin{equation*}
		a_{-4} = a_4 = \frac{1}{2}, a_{-3} = j, a_3 = -j
	\end{equation*}

	\subsection*{د)}
	\paragraph*{}
	می‌توان مشاهده کرد که به جز زمانی که
	$t = 2m$
	باشد، سری داده شده جملات صفر تولید می‌کند. در نتیجه:

	\begin{equation*}
		x_4(t) = e^{j\left(\frac{2\pi}{7}\right)(\frac{t}{2})}
		= e^{j \left(\frac{\pi}{7}\right)t}
	\end{equation*}

	می‌توان فرض کرد
	$\omega_0 = \frac{\pi}{7}$.
	با این فرض ضرایب سری فوریه به صورت زیر مشخص می‌شوند (بقیه ضرایب صفر هستند).

	\begin{equation*}
		a_1 = 1
	\end{equation*}

	\section*{سوال ۲}
	\paragraph*{}
	با توجه به نمودار، می‌توان مشاهده کرد که
	$T_0 = 4$
	و
	$\omega_0 = \frac{2\pi}{4} = \frac{\pi}{2}$.
	با استفاده از رابطه آنالیز سری فوریه داریم

	\begin{align*}
		a_0 &= \frac{1}{4} \int_{-2}^{2} x(t) e^{-j(0)(\frac{\pi}{2})t} dt \\
		&= \frac{1}{4} \int_{-2}^{2} x(t) dt \\
		&= \frac{1}{4} \left(\int_{-2}^{-1} x(t) dt + \int_{-1}^{+1} x(t) dt + \int_{1}^{2} x(t) dt \right) \\
		&= \frac{1}{4} \left( 1 + (2 + \frac{2 \times 1}{2}) + 1\right) \\
		&= \frac{5}{4}
	\end{align*}

	\begin{align*}
		a_3 &= \frac{1}{4} \int_{-2}^{2} x(t) e^{-j(3)(\frac{\pi}{2})t} dt \\
		&= \frac{1}{4} \left(\int_{-2}^{-1} x(t) e^{-j(\frac{3\pi}{2})t} dt
		+ \int_{-1}^{0} x(t) e^{-j(\frac{3\pi}{2})t} dt
		+ \int_{0}^{1} x(t) e^{-j(\frac{3\pi}{2})t} dt
		+ \int_{1}^{2} x(t) e^{-j(\frac{3\pi}{2})t} dt \right) \\
		&= \frac{1}{4} \left(\int_{-2}^{-1} e^{-j(\frac{3\pi}{2})t} dt
		+ \int_{-1}^{0} (t+2) e^{-j(\frac{3\pi}{2})t} dt
		+ \int_{0}^{1} (-t+2) e^{-j(\frac{3\pi}{2})t} dt
		+ \int_{1}^{2} e^{-j(\frac{3\pi}{2})t} dt \right) \\
		&= \frac{1}{4} \left( \frac{2}{3\pi} (1+j)
		+ \int_{-1}^{0} (t+2) e^{-j(\frac{3\pi}{2})t} dt
		+ \int_{0}^{1} (-t+2) e^{-j(\frac{3\pi}{2})t} dt
		+ \frac{2}{3\pi} (1-j)
		 \right)
	\end{align*}

	با استفاده از انتگرال جزء به جزء داریم

	\begin{align*}
		u = (t+2) &\Rightarrow du = dt \\
		dv = e^{-j(\frac{3\pi}{2})t} dt &\Rightarrow
		v = \frac{2}{3\pi} j e^{-j(\frac{3\pi}{2})t}
	\end{align*}

	\begin{align*}
		\Rightarrow
		\int_{}^{} u dv &= uv - \int_{}^{} vdu \\
		&= \frac{2}{3\pi} (t+2) j e^{-j(\frac{3\pi}{2})t}
		- \int_{}^{} \frac{2}{3\pi} j e^{-j(\frac{3\pi}{2})t} du \\
		&= \frac{2}{3\pi} (t+2) j e^{-j(\frac{3\pi}{2})t}
		- \int_{}^{} \frac{2}{3\pi} j e^{-j(\frac{3\pi}{2})t} dt \\
		&= \frac{2}{3\pi} (t+2) j e^{-j(\frac{3\pi}{2})t}
		+ \frac{4}{9\pi^2} e^{-j(\frac{3\pi}{2})t}
	\end{align*}

	\begin{align*}
		\Rightarrow
		\int_{-1}^{0} (t+2) e^{-j(\frac{3\pi}{2})t} dt
		&= \left(\frac{4}{3\pi} j + \frac{4}{9\pi^2}\right)
		- \left(\frac{2}{3\pi} + \frac{4}{9\pi^2}(-j)\right) \\
		&= \left(\frac{4}{9\pi^2} - \frac{2}{3\pi}\right)
		+ \left(\frac{4}{3\pi} + \frac{4}{9\pi^2}\right)j
	\end{align*}

	به طور مشابه
	\begin{align*}
		\int_{0}^{1} (-t+2) e^{-j(\frac{3\pi}{2})t} dt
		&= \left(\frac{2}{3\pi} - \frac{4}{9\pi^2}\right)
		+ \left(-\frac{4}{3\pi} - \frac{4}{9\pi^2}\right)j
	\end{align*}

	در نهایت
	\begin{align*}
		a_3 &=
		\frac{1}{4} \left( \frac{2}{3\pi} (1+j)
		+ 0
		+ \frac{2}{3\pi} (1-j)
		\right) \\
		&= \frac{1}{3\pi}
	\end{align*}

	\section*{سوال ۳}
	\subsection*{الف)}
	\paragraph*{}
	ضرایب سری فوریه با انجام مقیاس زمانی تغییر نمی‌کند. صرفا مقدار
	$\omega_0$
	تغییر می‌کند و دو برابر می‌شود.
	\begin{equation*}
		\omega_{0_2} = 2\omega_{0_1} = 2 \times \frac{2\pi}{2} = 2\pi
	\end{equation*}
	\begin{equation*}
		b_k = c_k
	\end{equation*}

	\subsection*{ب)}
	\begin{align*}
		x(t) &= \sum_{k=-\infty}^{\infty} c_k e^{jk\omega_0t} \\
		\Rightarrow
		x'(t) &= \sum_{k=-\infty}^{\infty} jk\omega_0 c_k e^{jk\omega_0t} \\
		&= \sum_{k=-\infty}^{\infty} b_k e^{jk\omega_0t} \\
		\Rightarrow
		b_k &= jk\omega_0 c_k = jk\pi c_k
	\end{align*}

	\subsection*{ج)}
	\paragraph*{}
	ضرایب سری فوریه در یک مقدار ثابت
	$e^{-jk\omega_0 t_0}$
	ضرب خواهند شد.

	\begin{equation*}
		b_k = e^{-jk\omega_0 t_0} c_k = e^{-jk (\pi) (\frac{1}{4})} c_k
		= e^{-j\frac{k\pi}{4}} c_k
	\end{equation*}

	\subsection*{د)}
	\paragraph*{}
	ضرایب سیگنال
	$\cos(2\pi t)$
	را
	$a_k$
	در نظر می‌گیریم. برای
	$a_k$
	داریم

	\begin{equation*}
		\cos(2\pi t) = \frac{e^{j(2)(\pi)t} + e^{-j(2)(\pi)t}}{2}
	\end{equation*}

	برای یکی بودن با
	$x(t)$،
	مقدار
	$\omega_0$
	را برابر
	$\pi$
	فرض می‌کنیم. در نتیجه مقادیر
	$a_k$
	به صورت زیر خواهد بود (بقیه ضرایب برابر صفر هستند).

	\begin{equation*}
		a_2 = a_{-2} = \frac{1}{2}
	\end{equation*}

	حال بنا بر خاصیت مدولاسیون سری فوریه داریم
	\begin{align*}
		b_k &= a_k * c_k \\
		&= \sum_{l=-\infty}^{\infty} a_l c_{k-l}
		= \frac{1}{2} (c_{k-2} + c_{k+2}) \\
	\end{align*}

	\section*{سوال ۴}
	\paragraph*{}
	می‌توان دوره تناوب هر دو سیگنال
	$x(t)$
	و
	$z(t)$
	را برابر ۱ در نظر گرفت (دوره تناوب $z(t)$ در شکل مشخص است).

	\begin{equation*}
		x(t+1) = 2\cos(2\pi t + 2\pi) + \sin(4\pi t + 4\pi)
		= 2\cos(2\pi t) + \sin(4\pi t) = x(t)
	\end{equation*}

	\begin{align*}
		x_T(t) * z_T(t) &= z_T(t) * x_T(t) \\
		&= \int_{-\frac{1}{2}}^{\frac{1}{2}} z_T(\tau)x_T(t-\tau) d\tau \\
		&= \int_{-\frac{1}{4}}^{\frac{1}{4}} x_T(t-\tau) d\tau \\
		&= \int_{-\frac{1}{4}}^{\frac{1}{4}} 2\cos(2\pi (t-\tau)) + sin(4\pi (t- \tau)) d\tau \\
		&= \frac{2}{\pi} \cos(2\pi t)
	\end{align*}

	\section*{سوال ۵}

	\begin{align*}
		\frac{1}{T} \int_{T}^{} c_n e^{jn\omega_0t} \times f(t) dt
		&= \frac{1}{T} \int_{T}^{} c_n e^{jn\omega_0t} \times
		\sum_{k=-\infty}^{\infty} c_k e^{jk\omega_0t} dt \\
		&= \frac{1}{T} \sum_{k=-\infty}^{\infty} \int_{T}^{} c_n e^{jn\omega_0t} \times c_k e^{jk\omega_0t} dt
	\end{align*}

	از خاصیت تعامد در سری فوریه نتیجه می‌گیریم

	\begin{align*}
		&= \frac{1}{T} \int_{T}^{} c_n e^{jn\omega_0t} \times c_{-n} e^{-jn\omega_0t} dt \\
		&= \frac{c_n \times c_{-n}}{T} \int_{T}^{} e^{jn\omega_0t} \times e^{-jn\omega_0t} dt \\
		&= \frac{c_n \times c_{-n}}{T} \int_{T}^{} dt \\
		&= c_nc_{-n} \\
	\end{align*}

	چون سیگنال حقیقی است
	\begin{align*}
		&= c_n c^{*}_n \\
		&= |c_n|^2
	\end{align*}

	\section*{سوال ۶}

	\begin{align*}
		\frac{1}{T} \int_{T}^{} |f(t)|^2 dt
		&= \frac{1}{T} \int_{T}^{} f(t)f^*(t) dt\\
		&= \frac{1}{T} \int_{T}^{} f(t) \sum_{k=-\infty}^{\infty} c_k e^{-jk\omega_0t} dt \\
		&= \sum_{k=-\infty}^{\infty} \frac{1}{T} c_k \int_{T}^{} f(t) e^{-jk\omega_0t} dt \\
		&= \sum_{k=-\infty}^{\infty} \frac{1}{T} c_k c_{-k} \\
		&= \frac{1}{T} \sum_{k=-\infty}^{\infty} c_k c^*_{k} \\
		&= \frac{1}{T} \sum_{k=-\infty}^{\infty} |c_k|^2
		\le \frac{1}{T} \sum_{k=-m}^{m} |c_k|^2 (\forall m) \\
		\Rightarrow
		\frac{1}{T} \int_{T}^{} |f(t)|^2 dt
		&\le \frac{1}{T} \sum_{k=-m}^{m} |c_k|^2 (\forall m)
	\end{align*}
\end{document}