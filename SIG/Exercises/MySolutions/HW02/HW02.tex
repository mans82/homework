\documentclass{article}

\usepackage{amsmath}
\usepackage{xepersian}

\settextfont{XB Niloofar}

\title{سیگنال‌ها و سیستم‌ها - دکتر سلیمی‌بدر}
\author{امیرحسین منصوری - ۹۹۲۴۳۰۶۹}
\date{تمرین سری ۲}

\begin{document}
	\maketitle

	\section*{سوال ۱}
	\paragraph*{}
	با توجه به بسط تیلور داریم:

	\begin{align*}
		e^x &= \sum_{n = 0}^{\infty} \frac{x^n}{n!} \\
		\Rightarrow
		e^{j \theta} &= \sum_{n = 0}^{\infty} \frac{(j \theta)^n}{n!} \\
		&= 1 + \frac{j \theta}{1!} + \frac{(j \theta)^2}{2!} + \frac{(j \theta)^3}{3!}
		+ ... \\
		&= 1 + \frac{j \theta}{1!}
		- \frac{\theta ^ 2}{2!}
		- \frac{j \theta ^ 3}{3!}
		+ \frac{\theta ^ 4}{4!}
		+ \frac{j \theta ^ 5}{5!}
		+ ... \\
		&= \left(1 - \frac{\theta ^ 2}{2!} + \frac{\theta ^ 4}{4!} - ...\right)
		+ j\left(\theta - \frac{\theta ^ 3}{3!} + \frac{\theta ^ 5}{5!} - ...\right) \\
		&= \cos(\theta) + j \sin(\theta)
	\end{align*}

	\section*{سوال ۲}
	\subsection*{الف)}
	\paragraph*{}
	با توجه به این که سیگنال شامل مساحتی غیر صفر است که بی‌نهایت بار تکرار شده است، انرژی سیگنال به بی‌نهایت میل می‌کند.

	\paragraph*{}
	با توجه به این که سیگنال متناوب است، توان کل سیگنال برابر توان در یکی از دوره‌های سیگنال خواهد بود:

	\allowdisplaybreaks
	\begin{align*}
		P(y(t)) &= \frac{1}{T} \int_{-\frac{T}{2}}^{+\frac{T}{2}} \left|y(\tau)\right|^2 d \tau \\
		&= \frac{1}{4} \int_{-2}^{+2} \left|y(\tau)\right|^2 d \tau \\
		&= \frac{1}{4} \left( \int_{-1}^{0} \left| y(\tau)\right|^2 d \tau
		+ \int_{0}^{1} \left|y(\tau)\right|^2 d \tau \right)\\
		&= \frac{1}{4} \left( \int_{-1}^{0} (\tau+1)^2 d \tau
		+ \int_{0}^{1} (-\tau+1)^2 d \tau \right) \\
		&= \frac{1}{4} \left((\frac{1}{3}) + (\frac{1}{3})\right) \\
		&= \frac{1}{6}
	\end{align*}

	\subsection*{ب)}
	\paragraph*{}
	انرژی سیگنال برابر است با:
	\allowdisplaybreaks
	\begin{align*}
		E(x(t)) &= \int_{- \infty}^{\infty} |x(\tau)|^2 d\tau \\
		&= \int_{- \infty}^{\infty} e^{-6\tau} u(\tau) d \tau \\
		&= \int_{0}^{\infty} e^{-6 \tau} d \tau \\
		&= \left[- \frac{1}{6} e^{6 \tau} \right]^{\infty}_{0} \\
		&= -\frac{1}{6} \left(0 - 1\right) \\
		&= \frac{1}{6}
	\end{align*}

	\paragraph*{}
	با توجه به متناهی بودن انرژی سیگنال، توان سیگنال برابر صفر خواهد بود:

	\begin{align*}
		P(x(t)) &= \lim_{T \rightarrow \infty} \frac{1}{2T} \int_{-T}^{+T} |x(\tau)|^2 d\tau \\
		&= \lim_{T \rightarrow \infty} \frac{1}{2T} \int_{0}^{+T} e^{-6 \tau} d\tau \\
		&= \lim_{T \rightarrow \infty} \frac{1}{2T} (-\frac{1}{6} \left[e^{-6 \tau}\right]^{T}_{0}) \\
		&= \lim_{T \rightarrow \infty} -\frac{1}{12T} (e^{-6T} - 1) \\
		&= -0(0 - 1) \\
		&= 0
	\end{align*}

	\section*{سوال ۳}
	\subsection*{۱)}
	\paragraph*{}
	فرض می‌کنیم
	$x_1(t) = t^2$
	و
	$x_2(t) = (t-1)^2$.
	در این صورت داریم:
	\begin{align*}
		x_1(t) &\rightarrow y_1(t) =
		\begin{cases}
			\begin{matrix}
				t^2 & t^2 < t^2 - 4t + 4 \\
				(t-1)^2 & t^2 \ge t^2 - 4t + 4
			\end{matrix}
		\end{cases} \\
		\Rightarrow
		x_1(t) &\rightarrow y_1(t) =
		\begin{cases}
			\begin{matrix}
				t^2 & t < 1 \\
				(t-1)^2 & t \ge 1
			\end{matrix}
		\end{cases} \\
	\end{align*}

	همچنین برای
	$x_2(t)$:
	\begin{align*}
		x_2(t) &\rightarrow y_2(t) =
		\begin{cases}
			\begin{matrix}
				(t-1)^2 & t < 2 \\
				(t-2)^2 & t \ge 2
			\end{matrix}
		\end{cases}
	\end{align*}

 	در نتیجه داریم:
	\begin{equation*}
		x_3(t) = x_1(t) + x_2(t) = t^2 + (t-1)^2
	\end{equation*}

	و:

	\begin{align*}
		x_3(t) \rightarrow y_3(t) &=
		\begin{cases}
			\begin{matrix}
				t^2 + (t-1)^2 & 2t^2 - 2t + 1 < (t-2)^2 + (t-3)^2 \\
				(t-1)^2 + (t-2)^2 & 2t^2 - 2t + 1 \ge (t-2)^2 + (t-3)^2
			\end{matrix}
		\end{cases} \\
		&=
		\begin{cases}
			\begin{matrix}
				t^2 + (t-1)^2 & t < 1.5 \\
				(t-1)^2 + (t-2)^2 & t \ge 1.5
			\end{matrix}
		\end{cases}
	\end{align*}

	اما:
	\begin{align*}
		y_1(t) + y_2(t) &=
		\begin{cases}
			\begin{matrix}
				t^2 + (t-1)^2 & t < 1 \\
				2(t-1)^2 & 1 \le t < 2 \\
				(t-1)^2 + (t-2)^2 & t > 2
			\end{matrix}
		\end{cases} \\
		& \ne y_3(t)
	\end{align*}

	در نتیجه سیستم خاصیت جمع‌پذیری ندارد و در نتیجه خطی نیست.

	\subsection*{۳)}
	\paragraph*{}
	سیستم خاصیت همگنی دارد:

	\begin{align*}
		x_1(t) \rightarrow y_1(t) &=
		\begin{cases}
			\begin{matrix}
				tx_1(t+3) & t < 0 \\
				4\cos(t^2)x_1(\sqrt t) & t \ge 0
			\end{matrix}
		\end{cases} \\
		\Rightarrow
		x_2(t) &= \alpha x_1(t) \rightarrow y_2(t) =
		\begin{cases}
			\begin{matrix}
				tx_2(t+3) & t < 0 \\
				4\cos(t^2)x_2(\sqrt t) & t \ge 0
			\end{matrix}
		\end{cases} \\
		&= \begin{cases}
			\begin{matrix}
				\alpha t x_1(t+3) & t < 0 \\
				4 \alpha \cos (t^2) x_1(\sqrt t) & t \ge 0
			\end{matrix}
		\end{cases} \\
		&= \alpha y_1(t)
	\end{align*}

	سیستم همچنین خاصیت جمع‌پذیری دارد:
	\begin{align*}
		x_1(t) \rightarrow y_1(t) &=
		\begin{cases}
			\begin{matrix}
				tx_1(t+3) & t < 0 \\
				4\cos(t^2)x_1(\sqrt t) & t \ge 0
			\end{matrix}
		\end{cases} \\
		x_2(t) \rightarrow y_2(t) &=
		\begin{cases}
			\begin{matrix}
				tx_2(t+3) & t < 0 \\
				4\cos(t^2)x_2(\sqrt t) & t \ge 0
			\end{matrix}
		\end{cases} \\
		\Rightarrow
		x_3(t) = x_1(t) + x_2(t) \rightarrow y_3(t) &=
		\begin{cases}
			\begin{matrix}
				tx_3(t+3) & t < 0 \\
				4\cos(t^2)x_3(\sqrt t) & t \ge 0
			\end{matrix}
		\end{cases} \\
		&=
		\begin{cases}
			\begin{matrix}
				tx_1(t+3) + tx_2(t+3) & t < 0 \\
				4\cos(t^2)x_1(\sqrt t) + 4\cos(t^2)x_2(\sqrt t) & t \ge 0
			\end{matrix}
		\end{cases} \\
		&= y_1(t) + y_2(t)
	\end{align*}

	در نتیجه سیستم خطی است.

	\subsection*{۴)}
	\paragraph*{}
	سیستم به وضوح علّی نیست. زیرا به عنوان مثال
	$y(-2) = x(-1)$.
	در نتیجه خروجی سیستم وابستگی به آینده دارد.


	\paragraph*{}
	برای بررسی تغییرپذیری با زمان، داریم:
	\begin{align*}
		x_1(t) &\rightarrow y_1(t) = x_1(\frac{t}{2}) \\
		x_2(t) = x_1(t - t_0) &\rightarrow y_2(t) = x_2(\frac{t}{2}) = x_1(\frac{t}{2} - t_0) \\
		y_1(t-t_0) &= x_1(\frac{t - t_0}{2}) \ne y_2(t)
	\end{align*}

	درنتیجه سیستم تغییرپذیر با زمان است.

	\subsection*{۵)}
	\paragraph*{}

	سیستم علّی است. زیرا برای محاسبه
	$y(t)$
	تنها به
	$x(t-2)$
	نیاز داریم و به آینده وابستگی نداریم. مقدار
	$u(t+2)$
	نیز به ازای هر
	$t$
	مشخص است.
	\paragraph*{}

	برای بررسی تغییرپذیری با زمان، داریم:
	\begin{align*}
		x_1(t) &\rightarrow y_1(t) = x_1(t - 2)u(t+2) \\
		x_2(t) = x_1(t - t_0) &\rightarrow y_2(t) = x_2(t-2)u(t+2) = x_1(t-2-t_0)u(t+2) \\
		\Rightarrow y_1(t-t_0) &= x_1(t-2-t_0)u(t+2-t_0) \ne y_2(t)
	\end{align*}

	در نتیجه سیستم تغییرپذیر با زمان است.

	\subsection*{۶)}
	\paragraph*{}
	داریم:
	\begin{equation*}
		y(t) = \int_{- \infty}^{-t} x(- \tau) d \tau = \int_{t}^{\infty} x(\tau) d\tau
	\end{equation*}

	چون انتگرال را تا بی‌نهایت محاسبه می‌کنیم، پس به ازای هر $t$ به مقدار سیگنال در زمان‌های جلوتر از t نیاز داریم. درنتیجه سیستم علّی نیست.

	\paragraph*{}
	برای بررسی تغییرپذیری با زمان، داریم:

	\begin{align*}
		x_1(t) \rightarrow y_1(t) &= \int_{t}^{\infty} x_1(\tau) d\tau \\
		x_2(t) = x_1(t-t_0) \rightarrow y_2(t) &= \int_{t}^{\infty} x_2(\tau) d\tau
		= \int_{t}^{\infty} x_1(\tau - t_0) d\tau \\
		&= \int_{t - t_0}^{\infty} x_1(\tau) d\tau \\
		y_1(t-t_0) &= \int_{t-t_0}^{\infty} x_1(\tau) d\tau \\
		&= y_2(t)
	\end{align*}

	در نتیجه، سیستم تغییر ناپذیر با زمان است.

	\subsection*{۷)}
	\paragraph*{}
	سیستم وارون پذیر است و می‌توان سیستم زیر را به عنوان معکوس آن در نظر گرفت:

	\begin{equation*}
		y_i(t) =
		\begin{cases}
			\begin{matrix}
				y(t + 2) & t \ge 0 \\
				y(t - 2) & t < 0
			\end{matrix}
		\end{cases}
	\end{equation*}

	درواقع همه قسمت‌های سیگنال
	$x(t)$
	عینا در
	$y(t)$
	نیز وجود دارد و تنها به مقدار ثابتی جابه‌جا شده‌اند. با اعمال معکوس این جابه‌جایی، می‌توان سیگنال
	$x(t)$
	را از
	$y(t)$
	به دست آورد.


	\subsection*{۸)}
	\paragraph*{}
	سیستم وارون پذیر است و می‌توان سیستم زیر را به عنوان معکوس آن در نظر گرفت:

	\begin{equation*}
		y_i(t) =
		\begin{cases}
			\begin{matrix}
				+\sqrt{y(t)} & y(t) > 0 \\
				y(t) & y(t) \le 0
			\end{matrix}
		\end{cases}
	\end{equation*}

	\paragraph*{}
	در واقع اگر خروجی سیستم داده شده در سوال، مثبت باشد، می‌دانیم که سیگنال داده شده نیز مثبت بوده است و از ضابطه اول استفاده شده است. در نتیجه برای به دست آوردن سیگنال ورودی (یا همان $x(t)$)، کافیست مقدار مثبت ریشه دوم سیگنال خروجی (یا همان $y(t)$) را در نظر بگیریم.

	\paragraph*{}
	همچنین اگر سیگنال خروجی منفی باشد، می‌دانیم که از ضابطه دوم استفاده شده و بنابراین سیگنال خروجی، همان سیگنال ورودی است.

	\subsection*{۹)}
	به ازای
	$x(t) = \sin^2(t)$،
	خواهیم داشت
	$y(t) = \frac{\sin(\sin^2(t) + 2t)}{\sin^2(t-1)}$
	و بنابراین
	$ |x(t)| \le 1$.
	اما داریم
	\begin{equation*}
		\lim_{t \rightarrow 1} y(t) = \lim_{t \rightarrow 1} \frac{\sin(\sin^2(t) + 2t)}{\sin^2(t-1)} = \infty
	\end{equation*}

	در نتیجه سیگنال پایدار نیست.

	\subsection*{۱۰)}
	به ازای
	$x(t) = \sin(t)$،
	داریم
	$|x(t)| < 1 $
	و
	\begin{align*}
		y(t) &=
		\begin{cases}
			\begin{matrix}
				t\sin(t) & t < |\sin(t)| \\
				\sin(-t) & t \ge |\sin(t)|
			\end{matrix}
		\end{cases} \\
		&=
		\begin{cases}
			\begin{matrix}
				t\sin(t) & t < 0 \\
				\sin(-t) & t \ge 0
			\end{matrix}
		\end{cases}
	\end{align*}

	اما داریم:
	\begin{equation*}
		\lim_{t \rightarrow -\infty} y(t) = \lim_{t \rightarrow -\infty} t\sin(t)
		= -\infty
	\end{equation*}

	در نتیجه سیستم پایدار نیست.

	\paragraph*{}
	برای بررسی علّی بودن، حالت‌های زیر را برای
	$y(t)$
	درنظر می‌گیریم:

	۱. شرط
	$t < |x(t)|$
	برقرار باشد: در این صورت
	$y(t) = tx(t)$
	که به وضوح وابستگی به آینده ندارد.

	۲. شرط
	$t \ge |x(t)|$
	برقرار باشد: در این صورت
	$t \ge 0$
	و
	$y(t) = x(-t)$
	و بنابراین خروجی باز هم وابستگی به آینده ندارد.

	در نتیجه، سیستم علّی است.
\end{document}